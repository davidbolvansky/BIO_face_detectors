\documentclass[]{article}
\usepackage[utf8]{inputenc}
\usepackage[slovak]{babel}
\usepackage[hidelinks]{hyperref}
\usepackage{graphicx}

\begin{document}
	
	\begin{titlepage}
		\begin{center}
			\textsc{{\LARGE Vysoké učení technické v~Brně\\[0.3em]
					Fakulta informačních technologií}}\\
			\vspace{\stretch{0.382}}
			{\Huge \textbf{Vplyv rozlíšenia a natočenia hlavy na detekciu tváre}\\[0.5em]}
			\Large{Biometrické systémy}
			\vspace{\stretch{0.618}}
		\end{center}
		
		{\noindent \large Lukáš Dekrét\\Dávid Bolvanský}
	\end{titlepage}
	
	\section{Úvod}
	Cieľom tohto projektu bolo preskúmať vplyv rozlíšenia a jasu na úspešnosť detekcie tváre. Bolo vybraných 5 nástrojov na detekciu tvárí, ktoré sú momentálne najpoužívanejšie a najpopulárnejšie. Na vybranom datasete fotografií sme vykonali niekoľko experimentov, kde sme skúšali meniť rozlíšenie/jas fotografií a sledovali sme, aký vplyv má táto zmena na detektory a ich úspešnosť v detekcií tvárí.
	
	\section{Dataset a výber fotiek na experimenty}
	Na účely experimentovania sme si vybrali dataset \textit{FBBD} \footnote{\url{http://vis-www.cs.umass.edu/fddb/}}. Z tohto datasetu sme vybrali 100 fotiek, na ktorých sa nachádzala len jedna tvár, a táto tvár mala fixnú pixelovú šírku tváre \-- 90 pixelov. Výber fotiek mal takéto obmedzenia hlavne z dôvodu experimentovania a stanovania si základných parametrov, ktoré fotografie budú mať. Použitie fixnej pixelovej dĺžky tvárí bolo odporúčané najmä pre experimenty zo zmenami rozlíšenia.
	V našom výbere fotiek na experimentovanie sme snažili mať čo najviac vyvážené rasové rozloženie, no kvôli prechádzajúcimi obmedzeniam a nedostatku takýchto fotiek toto rovnomerné rozloženie nebolo možné úplne dosiahnuť. Výsledných 100 fotiek tak obsahuje 72 osôb europoidnej rasy, 14 osôb mongoloidnej rasy a 14 osôb negroidnej rasy. Pre účely zisťovania \textit{true negative} prípadov sme si našli 14 ďalších fotografií, ktoré sa podobajú ľudskej tvári.
	
	\section{Výber nástrojov na detekciu tvárí}
	
	Vybrali nasledovných 5 nástrojov s ktorými sme následne prevádzali experimenty a zisťovali ako obstoja v netradičných / menej ideálnych podmienkach.
	
	\subsection{MTCNN}
	Momentálnu \textit{state-of-the-art} detekciu tvári je možné dosiahnuť pomocou \textit{Multi-task Cascade Convolutional Neural Network}. Táto metóda bola predstavená vo vedeckom článku \textit{Joint Face Detection and Alignment using Multi-task Cascaded Convolutional Networks} \footnote{\url{https://arxiv.org/pdf/1604.02878.pdf}}. Hotovú implementáciu (ktorú sme aj použili na experimenty) je možné nájsť na: \url{https://github.com/ipazc/mtcnn}.
	
	\section{Vyhodnotenie výsledkov, záver}
	
	
\end{document}