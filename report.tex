\documentclass[]{article}
\usepackage[utf8]{inputenc}
\usepackage[slovak]{babel}
\usepackage[hidelinks]{hyperref}
\usepackage{graphicx}

\begin{document}
	
	\begin{titlepage}
		\begin{center}
			\textsc{{\LARGE Vysoké učení technické v~Brně\\[0.3em]
					Fakulta informačních technologií}}\\
			\vspace{\stretch{0.382}}
			{\Huge \textbf{Vplyv rozlíšenia a natočenia hlavy na detekciu tváre}\\[0.5em]}
			\Large{Biometrické systémy}
			\vspace{\stretch{0.618}}
		\end{center}
		
		{\noindent \large Lukáš Dekrét\\Dávid Bolvanský}
	\end{titlepage}
	
	\section{Úvod}
	V~našej práci sme modelovali a simulovali\footnote{\url{https://www.fit.vutbr.cz/study/courses/IMS/public/prednasky/IMS.pdf}, strana 8} proces výroby oceľových rúr zo šrotu. Na základe modelu\footnote{\url{https://www.fit.vutbr.cz/study/courses/IMS/public/prednasky/IMS.pdf}, strana 7} a simulačných experimentov bude ukázané, ako veľmi sa zmení produkcia prevádzky, keď sa zmení sortiment, zmenia prestoje alebo zvýši kadencia.
	
	Zmyslom štúdie je zistiť, ako veľmi ovplyvní vybraný sortiment rúr produktivitu, efektivitu, a teda i celý zárobok firmy. Podľa slov výrobného riaditeľa firmy, železiarne by ocenili vytvorenie takejto simulačnej štúdie, a preto sme sa ubrali týmto smerom.
	
	Pre spracovanie modelu bolo nutné získať potrebné údaje pracovníkov, ktorí vo firme robia, technológov v~danej oblasti a preštudovať dlhé príručky ohľadom samotného výrobného procesu a rôznych sortimentov rúr. Firma poskytuje širokú škálu dĺžok rúr, šírok stien a rôznych akostí a preto bolo veľmi dôležité si zvoliť jeden typ rúr, ktorý bude typický pre danú firmu a bude odpovedať istému priemeru celého sortimentu.
	
	
\end{document}